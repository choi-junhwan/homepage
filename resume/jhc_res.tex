% LaTeX resume using res.cls
\documentclass[margin,centered]{res}
%\usepackage{helvetica} % uses helvetica postscript font (download helvetica.sty)
%\usepackage{newcent}   % uses new century schoolbook postscript font 
\usepackage{hyperref}
\setlength{\textwidth}{5.1in} % set width of text portion

\begin{document}

% Center the name over the entire width of resume:
 \moveleft.5\hoffset\centerline{\large\bf Jun-Hwan Choi}
% Draw a horizontal line the whole width of resume:
 \moveleft\hoffset\vbox{\hrule width\resumewidth height 1pt}\smallskip
% address begins here
% Again, the address lines must be centered over entire width of resume:
% \moveleft.5\hoffset\centerline{Department of Astronomy University of Texas at Austin 2515 Speedway, Stop C1400 Austin, TX 78712-1205 }
% \moveleft.5\hoffset\centerline{\href{https://choi-junhwan.github.io/homepage}{Homepage} / \href{https://www.linkedin.com/in/junhwan-choi-48922b36/}{LinkedIn}}
 \moveleft.5\hoffset\centerline{\href{https://choi-junhwan.github.io/homepage}{https://choi-junhwan.github.io/homepage}}
 \moveleft.5\hoffset\centerline{choi.junhwan@gmail.com}
% \moveleft.5\hoffset\centerline{junhwan.choi@mg.thedataincubator.com}
% \moveleft.5\hoffset\centerline{(413) 219-6355 or (512) 471-3351}
% \moveleft.5\hoffset\centerline{(413) 219-6355}


\begin{resume}

\vspace{-0.25cm}
\section{Objectives} Data Scientist/ML Engineer with a background in computational Astrophysics.


\section{Data Science \\ Employment \\ \& Experiences}
{\bf Data Scientist at Walmart Nov. 2021 - Present} \\
- Developing next best action recommendation for Sam's Club memeber personalization using a response model and a renewal uplift model. \\
- Working on Sam's club membership renewal prediction model explainability. \\
- Developing a general purpose causal analysis framework. Applying the framework for marketing, CRM, and strategy use cases in Walmart. \\
- Developing and deploying W+ membership benefit recommendation system using a large-scale nearest neighbor search. W+ member churn analysis (model prediction and model explanation).\\
- Walmart online purchase basket analysis based on association rules with PySpark pipeline for PoC for the Contextual intelligence project.\\
{\bf Data Scientist at SparkCognition Nov. 2016 - Nov. 2021} \\
- Building predictive machine learning model with time series data to predict future events using Random Forest, Decision Tree Boosting, and Artificial Neural Network Methods (including CNN and RNN). It particularly focuses on anomaly detection and forecast.\\
- Unsupervised anomaly detection for future events using clustering algorithms, one-class SVM, t-sne, and variational autoencoder. \\
- Developing a AutoML/Neural Evolution Machine Learning Platform {\bf {\it Darwin}}.\\
{\bf Data Incubator Mar. - May 2016}  \\
- Highly competitive data science bootcamp: Python, SQL, Machine Learning (including NLP and Time Series), Data Visualization, MapReduce/Apache Spark, Capstone Project\\
% - capstone project: Consumer's Complaints Analysis ({\scriptsize \url{http://jhc-complaints.herokuapp.com}})\\

\section{Patents \& Publication}
\begin{enumerate}
\item[1] ``Diverse clustering of a data set (US20230112096A1)'' \\ \textbf{J. Choi}, T. McDONNELL, Y. Lan, K. D. Moore, \& C.-Y. Ho

\item[2] ``Execution of a genetic algorithm having variable epoch size with selective execution of a training algorithm (US11853893B2) '' \\ S. Andoni, K. D. Moore, E. M. Bonab, \& \textbf{J. Choi}

\item[3] ``Ensembling of neural network models (US11610131B2)'' \\ S. Andoni, K. D. Moore, E. M. Bonab, \textbf{J. Choi}, \& T. S. McDonnell 

\item[4] ``Execution of a genetic algorithm with variable evolutionary weights of topological parameters for neural network generation and training (US11106978B2)'' \\ S. Andoni, K. D. Moore, E. M. Bonab, \textbf{J. Choi}, \& E. O. Korman

\item[5] ``Automated model building search space reduction (US10657447B1)'' \\ T. S. McDonnell, S. Andoni, \textbf{J. Choi}, J. Goode, Y. Lan, K. D. Moore, \& G. Sellers

\item[6] ``Divide and conquer: neuroevolution for multiclass classification'' \\ T. McDonnell, ..., 2018, \textbf{J. Choi}, et al, Proceedings of the Genetic and Evolutionary Computation Conference, 474

\end{enumerate} 

\section{Computer \\ Skills} 
{\bf Languages \& Software:} Python (including Pytorch/Tensorflow), C/C++, Fortran, SQL, Matlab/Octave, R,\\
{\bf High Performance Computing Experience:} Develop and implement numerical simulations in national super computing facilities (TACC, NCSA, and OLCF). Cloud computing in GCP/AWS.

\section{Academic \\ Projects} {\it Published more than 20 peer review academic journals including 10 leading author.}\\
{\bf Post Doctoral Scholar in University of Texas Aug. 2013 - Nov. 2016} \\ 
{\it Reionization and Galaxy Formation in the Local Universe:} Performing data analysis for the large cosmological simulation for the early Universe.\\
{\bf Post Doctoral Scholar in University of Kentucky Jul. 2010 - Jul. 2013 } \\
{\it The Early Massive Black Holes:} Implementing massively parallelized N-body/Hydrodynamics simulations and developing the data analysis (regression and power spectrum) to investigate a new channel of black hole formation in the early Universe.\\
{\bf Post Doctoral Scholar in UNLV Sep. 2007 - Jun. 2010} \\
{\it Galaxy formation in early Universe}  Developing massively parallelized cosmological N-body/Hydro simulation to study evolution of early galaxies.

\section{Other \\ Experiences} 
Co-instructor, who organizes, mentors, and lectures,  for {\it Freshman Research Initiative} course in Department of Astronomy, University of Texas Austin, Jan. - Dec. 2014  \\
Organizer for Astronomy Journal Club, UNLV, 2008 - 2009 \\
Refereed Papers: MNRAS, ApJ (2013 - )

\section{Computer \\ Skills} 
{\bf Languages \& Software:} Python, C/C++, Fortran, SQL, Matlab/Octave, R, Scala\\
{\bf Python Data Science Tools:} Scipy, Pandas, Matplotlib, StatsModels, Scikit-Learn, Keras/Tensorflow, Pytorch, EconML \\
{\bf High Performance Computing Experience:} Develop and implement numerical simulations in national super computing facilities such as TACC, NCSA, and OLCF.

\section{EDUCATION} 
Ph.D. in Astronomy, University of Massachusetts at Amherst (MA, USA), Aug. 2007 \\
M.S. in Astronomy, Yonsei University (Seoul, Korea), Feb. 1999 \\
B.S. in Astronomy (minor in Physics), Yonsei University (Seoul, Korea), Feb. 1997 \\



\end{resume}
\end{document}



