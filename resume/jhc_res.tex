% LaTeX resume using res.cls
\documentclass[margin,centered]{res}
%\usepackage{helvetica} % uses helvetica postscript font (download helvetica.sty)
%\usepackage{newcent}   % uses new century schoolbook postscript font 
\usepackage{hyperref}
\setlength{\textwidth}{5.1in} % set width of text portion

\begin{document}

% Center the name over the entire width of resume:
 \moveleft.5\hoffset\centerline{\large\bf Junhwan Choi}
% Draw a horizontal line the whole width of resume:
 \moveleft\hoffset\vbox{\hrule width\resumewidth height 1pt}\smallskip
% address begins here
% Again, the address lines must be centered over entire width of resume:
% \moveleft.5\hoffset\centerline{Department of Astronomy University of Texas at Austin 2515 Speedway, Stop C1400 Austin, TX 78712-1205 }
\moveleft.5\hoffset\centerline{choi.junhwan at gmail.com}
% \moveleft.5\hoffset\centerline{choi.junhwan@gmail.com}
% \moveleft.5\hoffset\centerline{junhwan.choi@mg.thedataincubator.com}
% \moveleft.5\hoffset\centerline{(413) 219-6355}


\begin{resume}

\vspace{-0.5cm}
\section{Objectives} Data Scientist with a background in computational Astrophysics.

\section{Data Science \\ Employment \\ \& Experiences}
{\bf Data Scientist at SparkCognition Nov. 2016 - Present} \\
- Building predictive machine learning model with time series data to predict future events using Random Forest, Decision Tree Boosting, and Artificial Neural Network Methods (including CNN and RNN). Unsupervised anomaly detection for future events using clustering algorithms, one-class SVM, t-sne, and variational autoencoder. \\
- Developing a AutoML/Neural Evolution Machine Learning Platform {\bf {\it Darwin}}.\\
{\bf Data Incubator Mar. - May 2016}  \\
- highly competitive data science bootcamp \\
- mini-projects: SQL, Machine Learning (including NLP and Time Series), Visualization, MapReduce, Apache Spark \\
- capstone project: Consumer's Complaints Analysis ({\scriptsize \url{http://jhc-complaints.herokuapp.com}})

\section{Papers  \\ \& Patents}

{\bf ``Divide and conquer: neuroevolution for multiclass classification''} \\ T. McDonnell, ..., \textbf{J. Choi}, et al, 2018, Proceedings of the Genetic and Evolutionary Computation Conference, 474 \\
{\bf ``Execution of a genetic algorithm with variable evolutionary weights of topological parameters for neural network generation and training (US20190080240A1)''} \\ S. Andoni, K. D. Moore, E. M. Bonab, \textbf{J. Choi}, \& E. O. Korman \\
{\bf ``Ensembling of neural network models (US10635978B2)''} \\ S. Andoni, K. D. Moore, E. M. Bonab, \textbf{J. Choi}, \& T. S. McDonnell

\section{Academic \\ Projects} Published more than 23 peer review academic journals including 10 leading author.\\
{\bf Post Doctoral Scholar in University of Texas Aug. 2013 - Nov. 2016} \\ 
{\it Reionization and Galaxy Formation in the Local Universe}\\
{\bf Post Doctoral Scholar in University of Kentucky Jul. 2010 - Jul. 2013 } \\
{\it The Early Massive Black Holes}: Improvement and implementation of massively parallelized N-body/Hydrodynamics simulations to investig\\
{\bf Post Doctoral Scholar in UNLV Sep. 2007 - Jun. 2010} \\
{\it Galaxy formation in early Universe:}

\section{Other \\ Experiences} 
Co-instructor, who organizes, mentors, and lectures,  for {\it Freshman Research Initiative} course in Department of Astronomy, University of Texas Austin, Jan. - Dec. 2014  \\
Organizer for Astronomy Journal Club, UNLV, 2008 - 2009 \\
Refereed Papers: MNRAS, ApJ (2013 - )

\section{Computer \\ Skills} 
{\bf Languages \& Software:} Python, C/C++, Fortran, SQL, Matlab/Octave, R, Scala\\
{\bf Python Data Science Tools:} Scipy, Pandas, Matplotlib, StatsModels, Scikit-Learn, Keras(Tensorflow/Theano), Pytorch \\
{\bf High Performance Computing Experience:} Develop and implement numerical simulations in national super computing facilities such as TACC, NCSA, and OLCF.

\section{EDUCATION} 
Ph.D. in Astronomy, University of Massachusetts at Amherst (MA, USA), Aug. 2007 \\
M.S. in Astronomy, Yonsei University (Seoul, Korea), Feb. 1999 \\
B.S. in Astronomy (minor in Physics), Yonsei University (Seoul, Korea), Feb. 1997 \\
 

\end{resume}
\end{document}



